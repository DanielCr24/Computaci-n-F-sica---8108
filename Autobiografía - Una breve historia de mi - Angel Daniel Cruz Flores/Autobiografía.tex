\documentclass[12pt, letterpaper]{article}
\usepackage[utf8]{inputenc}
\usepackage[spanish]{babel}
\usepackage[usename]{xcolor}


\title{Una breve historia de mi}
\author{Angel Daniel Cruz Flores }
\date{11 de Septiembre de 2022}

\begin{document}

\maketitle


\section{\tt{\LARGE{Academia}}}

\normalsize{
Para mi, como para muchos otros, la parte académica ha sido siempre una parte muy importante en mi vida, diria fundamental en la vida de cualquier ser humano, sin embargo, desde mi punto de vista está siendo mal aplicada, pues desde una edad temprana, la escuela fomenta la memorización por sobre el razonamiento y el pensamiento, dando así un sistema muy estricto, que mas allá de desarrollar en las personas la cualidad de pensar de manera propia y original,  que ayuden a la sociedad a mejorar, pero al mismo tiempo no olvidar la individualidad de cada estudiante. Al llevar a todos a un estándar único, las personas se ven privadas de originalidad. Por lo tanto, la escuela necesita educar a las personas capaces de pensar y actuar de manera independiente
}



    \subsection{\tt{\large{Pasado}}}
    
    \small{
    Mi vida académica comenzó a la edad de dos años y medio, cuando comencé preescolar en una escuela no muy lejos de mi casa, tan solo 10 minutos era lo que las separaba.\\
    
    Es importante mencionar que la escuela cuenta con preescolar y primaria en un mismo lugar, también tiene secundaria, sin embargo ésta se encuentra una calle atrás de primaria y secundaria.\\
    
    Curse preescolar, primaria y solo el primer año de secundaria en esa escuela, los otros dos los cursé en una escuela diferente, que se encontraba un poco más lejos, a 25 minutos de mi casa.\\
    
    Acabada la secundaria decidí realizar el examen Comipems, en el cual obtuve 122 aciertos, lo que me permitió ingresar a mi primera opción, que fue la Escuela Nacional Preparatoria No.9 ubicada en Insurgentes, no quedaba para nada cerca de mi casa, sin embargo ésto no fue un problema dado que solo fui 3 meses a clases presenciales y unos cuantos días a finales de 6to para realizar tutorías y actividades de integración, pues inició un paro indefinido que concluyó un mes antes que iniciara la pandemia por Covid-19, lo que nos obligó a tomar el resto del año, todo 5to y 6to, sin embargo, durante el breve periodo que asistí me trasladaba en carro o en metro, dado que yo vivo en Ecatepec, tardaba aproximadamente 40-50 minutos en trasladarme desde mi casa hasta la preparatoria y visceversa. 
    }
    
    \subsection{\tt{\large{Actualidad}}}
    
    \small{
    Hoy en día, me encuentro estudiando la licencietura en física en la Facultad de Ciencias en C.U. (UNAM), por lo dicho anterirormente, la Facultad queda muy lejos de mi casa; usualmente uso el metro para poder trasladarme de mi domicilio a la Facultad; subiendome en la estación Ecatepec, ubicada en la Línea B, después transbordo en la estación Oceanía para tomar la Línea 5 (Pantitlán - Politécnico), del mismo modo, transbordo en la estación La Raza con dirección Universidad. El tiempo aproximado que toma llegar de mi domicilio a C.U. usando la ruta descrita anteriormente es de una hora y media; uso la misma ruta solo que viceversa para trasladarme de la Facultad a mi domicilio.
    }
    
    
    \subsection{\tt{\large{¿Por qué Física?}}}
    
    \small{
    Desde pequeño supe que quería dedicar mi vida a la ciencia, realmente nunca hubo alguna otra carrera que me llamara la atención de la misma forma que la física lo hizo, decidí estudiar física por curiosidad, por querer entender como es que funciona todo, desde la partícula más pequeña hasta la galaxia más grande.\\
    
    Así mismo, elegí esta carrera por las incógnitas y preguntas sin respuesta e incluso las preguntas que jamás se ha hecho alguien, más allá de las cosas que ya se saben, lo que me atrajó mucho más la atención y me emociona es poder dar respuesta a todas esas preguntas, a generar nuevo conocimiento.\\
    }
    
    

\section{\tt{\LARGE{Sobre Hobbies y Pasatiempo}}}

\normalsize{
Ultimamente, derivado al inicio del semstre realmente no tengo mucho tiempo libre que dedicarle a mis pasatiempos o hobbies, sin embargo, en fines de semana e incluso cuando no tengo demasiada tarea que realizar, suelo ocupar mi tiempo en las siguientes actividades:
}


    \subsection{\tt{\large{Correr}}}
    
    \small{
    Derivado de la pandemia es que descubrí mi gusto por correr, es algo que disfruto mucho, me ayuda a pensar y a despejarme, cuando estoy agoviado poe cualquier cosa en general el correr es un escape a todo eso, así como que correr es un gran ejercicio, desde mi punto de vista es el mejor y aporta muchos beneficios al estilo de vida en general.\\
    
    Durante la pandemia y las vacaciones le dedicaba alrededor de 14 horas a la semana, sin embargo, desde el inicio del semestre, se redujo, por falta de tiempo, ahora le dedico un aproximado de 4 horas a la semana. 
    }
    
    
    \subsection{\tt{\large{Jugar videojuegos}}}
    
    \small{
    Del mismo modo, la pandemia hizo que descubriera ese hobbie o pasatiempo, al estar todo el día encerrado en mi habitación frente a una computadora fue más que obvio que buscara actividades que despejaran mi mente un rato, mi gusto por los videojuegos es gracias a plataformas como Twitch o YouTube.
    Antes de iniciar clases presenciales le dedicaba alrededor de 8 horas semanales, sin embargo, ahora solo le dedico un aproximado de 3 horas semanales. \\ 
    
    
Finalmente, no se si cuente como tal un pasatiempo o un hobbie, pero generalemente mis tiempos libres los paso con mi novia, ya sea pasando tiempo juntos de ida y de regreso a C.U., en nuestras horas libres y casi a diario haciendo videollamada en las noches. \\
}
    
    

\section{\tt{\LARGE{Sobre géneros de música}}}

\normalsize{
Realmente no podría definir mis dos géneros de música preferidas pues me gusta escuchar de todo un poco, sin embargo, si tuviera que elegir dos, elegiría los que más escucho. 
}


    \subsection{\tt{\large{Pop}}}
    
    \small{
    Suelo escuchar éste género de música generalemente cuando voy camino a la facultad y para hacer tarea, casi todo el día me la paso escuchando música, puse el género Pop en primer lugar pues la mayoría de canciones que escucho son de dicho género, así mismo, si tuviera que definir dos ejemplos de canciones y artistas elegiría a los siguientes: \\
    
    1. Harry Styles - Golden \\
    
    2. Dayglow - Can I call you tonight?
    }
    
    \subsection{\tt{\large{Urbano}}}
    
    \small{
    De igual forma suelo escuchar éste género de música cuando voy camino a la facultad y cuando realizo tareas, ya sean escolares o domésticas, cuando hago queaser, de comer, etc.; coloqué éste género en segundo lugar pues después del Pop, es el género q mas sueño escuchar, del mismo modo, si tuviera q definir dos ejemplos de canciones y artistas elegiría las siguientes: \\
    
    1. Quevedo: Bzrp Music Sessions, Vol. 52 \\

    2. Ojitos Lindos - Bad Bunny \\
    }


\section{\tt{\LARGE{Expectativas a futuro}}}

\normalsize{
En ésta sección hablaré tanto de las expectativas que tengo sobre la carrera de física en la UNAM, así como de mi plan de vida una vez acabada la carrera.
}

    \subsection{\tt{\large{¿Cuáles son mis expectativas de la Facultad de Ciencias?}}}
    
    \small{
    Realmente no tengo expectativas muy altas de la carrera de física en la Facultad de Ciencias, lo único que espero es poder obtener todo ese conocimiento necesario a lo largo de un año para poder cursar la carrera de física en otra institución, me interesaría realizarla en otro país. 
    
    Es decir, planeo cursar un año estifiando física en la Facultad de Ciencias y después solicitar una beca en otro país, del mismo modo, el perder un año no me molestaría, pues al estar aprendiendo no estoy perdiendo el tiempo, así como me sería mas facil obtener la beca gracias al año cursado en la UNAM.
    }
    
    \subsection{\tt{\large{Después de la carrera, ¿Qué pienso hacer?}}}

    \small{
    Como ya mencioné anteriormente, mi plan es estudiar física en el extranjero, después realizar una maestría y un doctorado, para finalmente dedicar mi vida a la ciencia y a la investigación y así generar nuevo conocimiento a partir de nuevas teorías.
    }
    
\section{\tt{\LARGE{Sección de Puntos Extras}}}

    \subsection{}
    
    
    \textit{\textbf{Antes de finalizar, me gustaría escribir un fragemnto de un poema que me gusta mucho:\\}}
    
    \textit{\textbf{“Deja que todo te suceda: la belleza y el terror. Solo sigue adelante. Ningún sentimiento es definitivo” - Rainer Maria Rilke}}
    
     
    \subsection{}
    

    {\textcolor{blue}{Por último, concluyo con una frase de Albert Einstein: }}
    
    {\textcolor{green}{"La mente es como un paracaídas, solo funciona si se abre" - Albert Einstein}}
    

\end{document}
