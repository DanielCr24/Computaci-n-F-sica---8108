\documentclass[a5paper, 11pt]{article}
\usepackage[utf8]{inputenc}
\usepackage[spanish]{babel}
\usepackage[top = 2.7cm, bottom = 2cm, left = 3cm, right = 2.5cm]{geometry}
\usepackage[dvipsnames]{xcolor}
\usepackage{fancyhdr}
\usepackage{graphicx}
\usepackage{enumerate}
\usepackage{caption}
\usepackage{float}




\pagestyle{fancy}
\lhead

\lfoot[\thepage]{\footnotesize{Angel Daniel Cruz Flores}}




\definecolor{mygreen}{RGB}{187, 252, 204}
\definecolor{myblue}{RGB}{187, 242, 252}
\definecolor{myred}{RGB}{252, 196, 187}
\definecolor{cafe}{rgb}{.94, .87, 8}

\pagecolor{cafe}
\begin{document}


\begin{center}

\vspace{2cm}

    \LARGE{\emph{La teoría del Todo}}\\[0.3cm]
    \small{Angel Daniel Cruz Flores}\\
    \small{\today}\\[1.2cm]

\end{center}


\section{\large{\emph{Una Breve historia del todo}}}

\small{
Esta sección, en general, abordará, de forma breve una reseña general de la película, así como los actores y productores de la película y de igual manera, una breve descripción de los personajes.
}

    \subsection{\emph{\normalsize{Acerca de la película.}}}
    
    \small{
    Pocos hombres podrán explotar el poder de la mente como lo ha hecho Stephen Hawking, un científico que además de ser uno de los más grandes astrofísicos de nuestros tiempos, también ha sido una persona capaz de hacer a un lado las barreras físicas que lo han mantenido anclado a una silla de ruedas, dependiente para desempeñar las tareas más básicas desde hace cincuenta años, apenas pudiendo controlar los movimientos de sus ojos y mejillas, para ser libre solo a través del pensamiento.\\
    
    Basada en 'Travelling to Infinity: My Life with Stephen', las memorias de Jane Hawking, esposa de Stephen Hawking, 'La teoria del todo' se atreve a adentrarse en la historia de una de las mentes más brillantes de nuestros tiempos, en su relación con su esposa y en cómo enfrenta una enfermedad degenerativa que no impide su brillantez y su necesaria aportación al mundo de la ciencia.\\
    
    Es también una historia de amor, pero no una de esas empalagosas o convencionales, nada en Hawking ha sido así. Los retos científicos que se propuso con la ciencia, también son vividos con su pareja Jane Wilde, con quien se conoció poco antes de conocer su diagnóstico, se casó y tuvo tres hijos. La relación entre ambos es la esencia de esta película: los desafíos diarios a superar por parte de Hawkings y los inmensos sacrificios a los que desde muy temprano se ve expuesta Wilde. Si es de los lectores que se permite reflexiones con las películas, La Teoría del Todo recuerda que la vida realmente no es fácil para muchos, pero aún así son capaces de llevar unas vidas amorosas y ejemplares.
    }
    
    \subsection{\emph{\normalsize{Cast}}}
    
    \small{
    
        \subsubsection{Actores Principales - Personajes importantes}
        
             \begin{enumerate}[1.]
    
                \item Eddie Redmayne - Stephen Hawking
                \item Felicity Jones - Jane Hawking
                \item Harry Lloyd - Brian
                \item David Thewlis - Dennis Sciama
                \item Charlie Cox - Jonathan Hellyer Jones
                \item Maxine Peake - Elaine Mason
    
            \end{enumerate}
    
        \subsubsection{Actores Secundarios - Personajes no tan importantes}
        
        \begin{enumerate}[1.]
    
                \item Abigail Cruttenden - Isobel Hawking
                \item Tom Prior - Robert Hawking
                \item Alice Orr-Ewing - Diana King
                \item Michael Marcus - Ellis
                \item Emily Watson - Beryl Wilde
                \item Guy Oliver-Watts - George Wilde
                \item Simon McBurney - Frank Hawking
                \item Christian McKay - Roger Penrose
    
            \end{enumerate}
        
        \subsubsection{Productores}
        
        \begin{enumerate}[1.]
    
                \item James Marsh - Director
                \item Tim Bevan, Lisa Bruce, Eric Fellner, Anthony McCarten - Productores
                \item Richard Hewitt - Coproductor
                \item Jóhann Jóhannsson - Compositor de la Banda Sonora
                \item Steven Noble - Vestuario
                \item Mark Holt - Supervisor de efectos visuales
                \item Deborah Saban - Encargado de sonido
                \item Sidony Etherton - Maquillaje
                \item Carrie-Ann Banner - Director de animación
                \item Mark Milsome - Cámara
                \item Benoit Delhomme - Director de fotografía
                \item Nina Gold - Director de Reparto
                \item Dan Malsen - Realizador/Guionista
    
            \end{enumerate}
    
    }
    
    
    
    
    \subsection{\emph{\normalsize{Un universo dentro de cada persona}}}
    
        \small{
    
        \begin{itemize}\renewcommand{\labelitemi}{$\propto$}
        
            \item \fcolorbox{myblue}{mygreen}{\textcolor{myred} {Stephen Hawking (Eddie Redmayne):}} La historia de 'La teoría del todo' se basa en la historia real del famoso astrofísico Stephen Hawking (interpretado por Eddie Redmayne), un estudiante de física brillante de la Universidad de Cabridge, rebelde, que al principio de la película se muestra desinteresado e irresponsable, cambia cuando se le es diagnosticado con Esclerosis Lateral Amiotrófica, quien, con apoyo de su esposa Jane Wilde logran superar todas las adversidades que ésta enfermedad trae consigo, para así convertirse en unos de los más grandes físicos en toda la historiaa de la humanidad.
            
            \item \fcolorbox{myblue}{mygreen}{\textcolor{myred} {Jane Hawking (Felicity Jones):}} Como ya se menciono anteriormente, la historia no solo se centra en Stephen Hawking, la historia gira alrededor de él y de su esposa, Jane Hawking (interpretada por Felicity Jones), una estudiante de letras francesas y españolas,  en la película, éste personaje se nos muestra como cariñosa y dispuesta a apoyar a su esposo en lo que sea necesario, además de una madre comprensiva y cariñosa, del mimso modo, al igual  que Stephen Hawking, Jane, posee una mente brillante, sin embargo, al final de la película se le puede ver exhausta y cansada debido a lo que conlleva ser esposa de Hawking. 
            
            \item \fcolorbox{myblue}{mygreen}{\textcolor{myred} {Brian (Harry Lloyd):}} Brian es el amigo y colega de Stephen durante su doctorado, al igual que Hawking, el es estudiante de física, irresponsable y un poco rebelde. 
            
            \item \fcolorbox{myblue}{mygreen}{\textcolor{myred} {Dennis Sciama (David ):}} Gran parte en la vida de Hawking fue su estadia en el doctorado, ahí, su profesor y guia fue el físico y gran cosmologo Dennis Sciama, quien lo ayudó durante todo este tiempo, sin duda un gran mentor para el, con una mente brillante y audaz.
            
            \item \fcolorbox{myblue}{mygreen}{\textcolor{myred} {Jonathan Hellyer Jones (Charlie Cox):}} Era el director de coro de una iglesia, a la cual asistía Jane de manera recurrente, pues se unió al coro, Jonathan se ofrece a ayudar a Jane a cuidar de Stephen, así como ser el profesor de piano de Robert, hijo de Jane y Stephen; al final, Jane decide separarse de Stephen y entablar una relación con Jonathan, sin embargo, Jane y Stephen continuaron con una relación de amistad. 
            
            \item \fcolorbox{myblue}{mygreen}{\textcolor{myred} {Elaine Mason (Maxine Peake):}} Es la enfermera de Stephen, alegre, divertida y siempre dispuesta a ayudar, finalmente se termina casando con Stephen.
            
        \end{itemize}
        
        \begin{figure}[H]
            
            \raggedleft
            \caption*{La Teoría del Todo.}
            \includegraphics[scale=0.24, angle=15]{La Teoría del Todo 3.jpg}
            
        \end{figure}
        
        }
        
        \newpage
    

\section{\emph{\large{¿Por qué la Teoría del Todo?}}}

\small {

    \subsection{Razones por las que elegí ésta película}
    
    Elegí ésta película pues es una de mis favoritas, es más, podría decir que es mi película favorita, en ella encontré una de mis grandes pasiones, la física, fue gracias a ella que decidí entrar a ésta carrera, del mismo modo, en ella hallo esperanza pues, a pesar de todas las dificultades o altibajos que hay en la vida, siempre hay una forma de salir adelante, lo importante es nunca dejar de pelear y trabajar por aquello que anhelamos, del mismo modo, me hizo darme cuenta que la vida sin alguien a quien amar, alguien que esté con nosotros o simplemente un sueño y una pasión no tendría mucho sentido.
    
    \subsection{\emph{\normalsize{Personaje que más me agradó}}}
    
        \begin{figure}[H]
            \raggedleft
            \includegraphics[scale=0.24, angle=-18]{La Teoría del Todo 1.jpg} 
            \caption{\raggedleft{Stephen y Jane. Ambos son mis personajes favoritos, pues la fortma en la que se complementaban me parece muy única.}}
            
        \end{figure}
    
    \subsection{\emph{\normalsize{Personaje que menos me agradó}}}
    
        \begin{figure}[H]
            \raggedleft
            \includegraphics[scale=0.4, angle=8]{f24dbc5c3fb9b44b309d8173e7.jpg}\caption{\raggedleft{Elaine: Su personalidad era agradable y divertida, sin embargo, siempre tuvo dobles intenciones.}}
            
            
        \end{figure}

\hspace{-2.4cm} {Una única ecuación}
\newline
\hspace{-2.4cm} {que lo explique}
\newline
\hspace{-2.4cm} {todo en el universo}

}

\end{document}
